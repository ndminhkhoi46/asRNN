\documentclass[10pt,a4paper,draft]{report}
\usepackage[utf8]{inputenc}
\usepackage{amsmath}
\usepackage{amsfonts}
\usepackage{amssymb}
\begin{document}
\section{Higher dimensional Derivative}
a) $y_i = \sum_{j}x_jW_{j,i}$ implies 
\begin{align*}
\frac{\partial y_i}{\partial W_{l,k}} =
	\begin{cases}
		0, \text{when } k \neq i\\
		x_l, \text{when } k = i
	\end{cases}
	= x_l
\end{align*}
Since index of $y$ coincides with the column index, the gradient is a row vector, while the partial derivative is a column vector. In other words, $\nabla_{W}(y) = x = \frac{\partial y}{\partial W}^\intercal$.
Given $f(y) \in \mathbb R$, $\frac{\partial f}{\partial W} = \frac{\partial f}{\partial y}x^\intercal$.
\newline
b)$y_i = \sum_{j}x_jW^\intercal_{j,i} = \sum_{j}x_jW_{i,j}$ implies 
\begin{align*}
\frac{\partial y_i}{\partial W_{l,k}} =
	\begin{cases}
		0, \text{when } l \neq i\\
		x_k, \text{when } l = i
	\end{cases}
	= x_k
\end{align*}
Since index of $y$ coincides with the row index, the gradient is a column vector, while the partial derivative is a column vector. In other words, $\nabla_{W}(y) = x^\intercal = \frac{\partial y}{\partial W}^\intercal$.
Given $f(y) \in \mathbb R$, $\frac{\partial f}{\partial W} = \frac{\partial f}{\partial W^\intercal}^\intercal = x\frac{\partial f}{\partial y}^\intercal$.
\newline
c)$y_i = \sum_{j}x_jW^{-1}_{j,i} = \sum_{j}x_j det(W) adj(W)_{j,i}$ implies 
\begin{align*}
\frac{\partial W^{-1}_{i,j}}{\partial W_{k,l}}
	\begin{cases}
	\end{cases}
\end{align*}
\newline
d)Let $W_1,W_2 \in\mathbb R^{n\times n}$. Then $y_i = \sum_{j}o_j(W_2^{-1})_{j,i} = \sum_{j}tanh(xW_1W_2)_j(W_2^{-1})_{j,i}$ implies 
\begin{align*}
\frac{\partial y_i}{\partial x_k} 
  &= \sum_{j}\frac{\partial y_i}{\partial o_j}\frac{\partial o_j}{\partial x_k}\\
  &= \sum_{j}(W_2^{-1})_{j,i}(1-tanh(xW_1W_2)_j^2)(W_2)_{k,j}
\end{align*}
In other words, $\frac{\partial y}{\partial x} = W_2^{-1} D W_1W_2, D = diag[I-tanh(xW_1)^2].$
\end{document}